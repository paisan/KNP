
%!TEX TS-program = xelatex
%!TEX encoding = UTF-8 Unicode

\chapter{ข้อมูลทั่วไป \\
General Information}

\section{วัตถุประสงค์และขอบเขตของคู่มือสนามบิน}

คู่มือปฏิบัติงานสนามบินฉบับนี้ มีจุดประสงค์ดังนี้
	\begin{enumerate}
	\item เพื่อใช้แสดงถึงวิธีปฏิบัติงานของสนามบินให้เกิดความปลอดภัยต่ออากาศยาน
	\item เพื่อใช้เป็นที่รวบรวมข้อมูลสนามบิน
	\item เพื่อใช้เป็นเอกสารอ้างอิงสำหรับการตรวจสอบการดำเนินงานสนามบินให้เป็นไปตามข้อกำหนดและมาตรฐานต่าง ๆ
\end{enumerate}

คู่มือฯนี้มีขอบเขต ดังนี้
วิธีการปฏิบัติงานสำหรับสนามบินขนงพระ เพื่อความปลอดภัยของอากาศยานที่ปฏิบัติการบินที่สนามบินขนงพระ เท่านั้น

\section{ข้อกฎหมายที่เกี่ยวข้องกับการขอและออกใบอนุญาตสนามบินส่วนบุคคล }

\begin{enumerate}
	\item พระราชบัญญัติการเดินอากาศ พ.ศ. 2497 ซึ่งแก้ไขเพิ่มเติมโดย พระราชบัญญัติการเดินอากาศ (ฉบับที่ 11) พ.ศ. 2551
	\item กฎกระทรวงว่าด้วยการขอและการออกใบอนุญาตจัดตั้งสนามบิน พ.ศ. 2550
	\item ประกาศสำนักงานการบินพลเรือนแห่งประเทศไทยเรื่องมาตรฐานคู่มือสนามบินส่วนบุคคล พ.ศ. 2561
\end{enumerate}

\section{เงื่อนไขการอนุญาตใช้สนามบิน (Conditions of Use)}

\subsection{รายละเอียดแสดงช่วงเวลาทำงานของสนามบินส่วนบุคคล}

สนามบินขนงพระ มีช่วงเวลาทำงานตั้งแต่เวลาประมาณ 6:00 น. ถึง 18:00 น. และมีช่วงเวลาเปิดใช้บริการแก่อากาศยานตั้งแต่พระอาทิตย์ขึ้น ถึง พระอาทิตย์ตก และเวลาอื่นนอกเหนือจากนี้โดยเปิดให้ใช้เป็นกรณีไปตามที่ร้องขอ  การนำอากาศยานมาขึ้นลงจะต้องได้รับอนุญาตจากผู้จัดการสนามบินหรือผู้ได้รับมอบหมายก่อนทุกครั้ง เนื่องจากเป็นสนามบินส่วนบุคคล \\
	
\noindent หมายเลขโทรศัพท์ติดต่อผู้จัดการสนามบิน 086 657 1510 

\subsection{รายละเอียดประประเภทของกฎการจราจรที่ต้องปฏิบัติ}

กฎการบินทั่วไป (GENERAL RULES) และ กฎการบินด้วยทัศนวิสัย (VFR)

\subsection{รายละเอียดประประเภททางวิ่งที่ให้บริการ}

ทางวิ่ง 10 ประเภท Non – Instrument Runway \\
ทางวิ่ง 28 ประเภท Non –Instrument Runway 

\subsection{รายละเอียดที่แสดงว่าสนามบินส่วนบุคคลไม่ได้ปิดให้บริการแก่บุคคลภายนอก}

\begin{enumerate}
	\item สนามบินขนงพระได้รับใบอนุญาตจัดตั้งเป็นสนามบินส่วนบุคคล และตามเงื่อนไขใบอนุญาต และวิธีปฏิบัติการของสนามบินขนงพระให้ใช้งานเป็นแบบสนามบินส่วนบุคคล
	\item รายละเอียดใบอนุญาตจัดตั้งเป็นสนามบินส่วนบุคคล  ตามแสดงในคู่มือฯนี้ \footnote{รายละเอียดในภาคผนวก ก. (\ref{ใบอนุญาตที่ขึ้นลงชั่วคราวอากาศยานขนงพระ})}
	 
\end{enumerate}

\subsection{ระบบการบันทึกข้อมูลของอากาศยาน (Recordings 0f Aircraft Movements) ที่แสดงถึงเที่ยวบินขาเข้าและขาออก}

\begin{enumerate}
	\item ก่อนทำการบินทุกครั้ง ผู้ทำการบินจะต้องยื่นแผนการบิน ( Flight Plan ) ไปยังหน่วยงานให้บริการจราจรทางอากาศตามรูปแบบของแผนการบิน ได้แก่ ข้อมูลของอากาศยานที่จะทำการบินมา ณ สนามบินขนงพระ ประกอบด้วย ชื่อนักบิน แบบอากาศยาน เครื่องหมาย สัญชาติ และทะเบียน เวลามาถึง/ออกจากสนามบิน
	\item ข้อมูลสถิติเที่ยวบินขาเข้าและขาออก  จะลงบันทึกในสมุดปูมของสนามบิน (Airfield Log) โดยมี รายละเอียดเกี่ยวกับเที่ยวบิน ประกอบด้วย ชื่อนักบิน แบบอากาศยาน เครื่องหมาย สัญชาติ และทะเบียน เวลามาถึง/ออกจากสนามบิน 
\end{enumerate}

\subsection{กรณีมีเที่ยวบินทำการบิน ณ สนามบินขนงพระ โดยไม่ได้รับอนุญาต}

ในกรณีที่พบว่ามีเที่ยวบินที่ไม่ได้รับอนุญาตทำการบิน ณ สนามบินขนงพระ โดยไม่ใช่เที่ยวบินที่มีเหตุจำเป็นฉุกเฉิน เจ้าหน้าที่จะรายงานข้อมูลเที่ยวบิน เช่น เครื่องหมายทะเบียน แบบอากาศยาน และชื่อนักบิน ไปยังสำนักงานการบินพลเรือนแห่งประเทศไทย

ระบบการบันทึกข้อมูลของอากาศยาน (Recordings 0f Aircraft Movements) ที่แสดงถึงเที่ยวบินขาเข้า

\section{หน้าที่และความรับผิดชอบของผู้ได้รับใบอนุญาตจัดตั้งเป็นสนามบินส่วนบุคคล (Private Aerodrome Operator)}

\begin{enumerate}
	\item ตามกฎกระทรวง การว่าด้วยการขอและการออกใบอนุญาตจัดตั้งสนามบิน พ.ศ. 2550  ผู้ขอจัดตั้งต้องมีความรู้ความสามารถในการจัดการสนามบินส่วนบุคคลตามที่อธิบดีประกาศกำหนด \\

	นายอนุทิน ชาญวีรกูล ได้รับอนุญาตให้จัดตั้งที่ขึ้นลงชั่วคราวของอากาศยานครั้งแรกตามใบอนุญาตเลขที่ 1535/2556 เมื่อวันที่ 17 ธันวาคม 2556 และได้รับการต่ออายุมาโดยตลอด รวมระยะเวลามากกว่า 4 ปี โดยครั้งล่าสุด ได้รับใบอนุญาตจัดตั้งฯ ที่ 654/2559 และ นายอนุทิน ชาญวีรกูล ได้ดำเนินการที่ขึ้นลงชั่วคราวฯ อย่างปลอดภัยมาโดยตลอด รวมทั้งปฏิบัติตามคำสั่งของทางราชการอย่างเคร่งครัด   จึงแสดงให้เห็นถึงว่า นายอนุทิน ชาญวีรกูล มีความรู้ความสามารถในการจัดการสนามบินส่วนบุคคล เป็นอย่างดี
	\item ตามประกาศสำนักงานการบินพลเรือนแห่งประเทศไทยเรื่องมาตรฐานคู่มือสนามบินส่วนบุคคล พ.ศ. 2561 \\
	นายอนุทิน ชาญวีรกูล ในฐานะเจ้าของสนามบินได้ดำเนินการและจักดำเนินการ ดังต่อไปนี้
	\begin{itemize}
		\item ได้จัดทำคู่มือสนามบินส่วนบุคคลตามประกาศฯ และยื่นต่อ สำนักงานการบิน
พลเรือนแห่งประเทศไทย ด้วยคู่มือฯ ฉบับนี้
		\item ปรับปรุงคู่มือฯ ให้มีข้อมูลเป็นปัจจุบัน เสมอ
		\item จัดทำรายละเอียดและแผนปฏิบัติการต่างๆ และปฏิบัติตามรายละเอียดนั้นๆ
		\item จัดหาบุคคลที่มีความรู้ความสามารถมาปฏิบัติงาน
	\end{itemize}
\end{enumerate}

นอกจากนี้ นายอนุทิน ชาญวีรกูล มีหน้าที่และความรับผิดชอบ ตามเงื่อนไขที่ สำนักงานการบินพลเรือนแห่งประเทศไทย จะกำหนดขึ้น และตามกฎหมายอื่นๆที่เกี่ยวข้อง

นายอนุทิน ชาญวีรกูล จะดำเนินงานสนามบินส่วนบุคคล ให้เกิดความปลอดภัยสูงสุด 
